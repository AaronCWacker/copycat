\documentclass[a4paper]{article}

%% Language and font encodings
\usepackage[english]{babel}
\usepackage[utf8x]{inputenc}
\usepackage[T1]{fontenc}

%% Sets page size and margins
\usepackage[a4paper,top=3cm,bottom=2cm,left=3cm,right=3cm,marginparwidth=1.75cm]{geometry}

%% Useful packages
\usepackage{listings}
\usepackage{amsmath}
\usepackage{graphicx}
\usepackage[colorinlistoftodos]{todonotes}
\usepackage[colorlinks=true, allcolors=blue]{hyperref}

\definecolor{lightgrey}{rgb}{0.9, 0.9, 0.9}
\lstset{ %
    backgroundcolor=\color{lightgrey}}

\title{Distributed Behavior in a Fluid Analogy Architecture}
\author{Lucas Saldyt, Alexandre Linhares}

\begin{document}
\maketitle

\begin{abstract}
    [Insert abstract]
\end{abstract}

%% This paper stems from Melanie Mitchell's (1993) and Douglas Hofstadter's \& FARG's (1995) work on the copycat program. 
%% This project focuses on effectively simulating intelligent processes through increasingly distributed decision-making.
%% In the process of evaluating the distributed nature of copycat, this paper also proposes a "Normal Science" framework. 

\section{Introduction}

    Copycat's behavior is based on the "Parallel Terraced Scan," a humanistic-inspired search algorithm.
The Parallel Terraced Scan corresponds to the psychologically-plausible behavior of briefly browsing, say, a book, and delving deeper whenever something sparks one's interest. 
The Parallel Terraced Scan is a mix between a depth-first and breadth-first search.
To switch between modes of search, FARG models use the global variable \emph{temperature}, which is ultimately a function of the rule strength and the strength of each structure in copycat's \emph{workspace}, another centralized structure.
However, it is not clear a global, unifying central structure like temperature is needed.
In fact, this structure may even be harmful to FARG architectures eventually.
This paper explores the extent to which copycat's behavior can be maintained while distributing decision making.

    Specifically, []

%% Specifically, this paper attempts different refactors of the copycat architecture.
%% First, the probability adjustment formulas based on temperature are changed.
%% Then, we experiment with two methods for replacing temperature with a distributed metric.
%% Initially, temperature is removed destructively, essentially removing any lines of code that mention it, simply to see what effect it has.
%% Then, a surgical removal of temperature is attempted, leaving in tact affected structures or replacing them by effective distributed mechanisms.
%% 
%% To evaluate the distributed nature of copycat, this paper focuses on the creation of a `normal science' framework.
%% By `Normal science,' this paper means the term created by Thomas Kuhn--the collaborative enterprise of furthering understanding within a paradigm. 
%% Today, "normal science" is simply not done on FARG architectures (and on most computational cognitive architectures too... see Addyman \& French 2012). 
%% Unlike mathematical theories or experiments, which can be replicated by following the materials and methods, computational models generally have dozens of particularly tuned variables, undocumented procedures, multiple assumptions about the users computational environment, etc.
%% It then becomes close to impossible to reproduce a result, or to test some new idea scientifically. 
%% This paper focuses on the introduction of statistical techniques, reduction of "magic numbers", improvement and documentation of formulas, and proposals for statistical human comparison.
%% 
%% We also discuss, in general, the nature of the brain as a distributed system.
%% While the removal of a single global variable may initially seem trivial, one must realize that copycat and other cognitive architectures have many central structures.
%% This paper explores the justification of these central structures in general.
%% Is it possible to model intelligence with them, or are they harmful?

%% {Von Neumann Discussion}
%% {Turing Completeness}
%% {Simulation of Distributed Processes}
%% {Efficiency of True Distribution}
%% {Temperature in Copycat}
%% {Other Centralizers in Copycat}
%% {The Motivation for Removing Centralizers in Coat}

\section{Methods}
    \subsection{Formula Documentation}
        Many of copycat's formulas use magic numbers and marginally documented formulas.
        This is less of a problem in the original LISP code, and more of a problem in the twice-translated Python3 version of copycat.
        However, even in copycat's LISP implementation, formulas have redundant parameters.
        For example, if given two formulas: $f(x) = 2x$ and $g(x) = x^2$, a single formula can be written $h(x) = 2x^2$ (The composed and then simplified formula).
        Ideally, the adjustment formulas within copycat could be reduced in the same way, so that much of copycat's behavior rested on a handful of parameters in a single location, as opposed to more than ten parameters scattered throughout the repository.
        Also, often parameters in copycat have little statistically significant effect.
        As will be discussed in the $\chi^2$ distribution testing section, any copycat formulas without a significant effect will be hard-removed.
    \subsection{Testing the Effect of Temperature}
        To begin with, the existing effect of the centralizing variable, temperature, will be analyzed.
        As the probability adjustment formulas are used by default, very little effect is had.
        To evaluate the effect of temperature-based probability adjustment formulas, a spreadsheet was created that showed a color gradient based on each formula.
        [Insert spreadsheet embeds]
        Then, to evaluate the effect of different temperature usages, separate usages of temperature were individually removed and answer distributions were compared statistically (See section: $\chi^2$ Distribution Testing).
    \subsection{Temperature Probability Adjustment}
        Once the effect of temperature was evaluated, new temperature-based probability adjustment formulas were proposed that each had a significant effect on the answer distributions produced by copycat.
        [Insert formula write-up]
    \subsection{Temperature Usage Adjustment}
        Once the behavior based on temperature was well understood, experimentation was made with hard and soft removals of temperature.
        First, a branch of the repository was created where all mentions of temperature were removed.
        [Insert nuke write-up]
        Then, a branch of the repository (the second revision of copycat to-be) was created, where temperature was removed surgically.
        [Insert surgical write-up]
    \subsection{$\chi^2$ Distribution Testing}
        To test each different branch of the repository, a scientific framework was created.
        Each run of copycat on a particular problem produces a distribution of answers.
        Distributions of answers can be compared against one another with a (Pearson's) $\chi^2$ distribution test.
        [Insert $\chi^2$ formula]
        [Insert $\chi^2$ calculation code snippets]
    \subsection{Effectiveness Definition}
        Quantitatively evaluating the effectiveness of a cognitive architecture is difficult.
        However, for copycat specifically, effectiveness can be defined as a function of the frequency of desirable answers and inverse frequency of undesirable answers.
        Since answers are desirable to the extent that they respect the original transformation of letter sequences, desirability can also be approximated by a concrete metric.
        A simple metric for desirability is simply the existing temperature formula, or some variant of it.
        So, a given version of copycat might be quantitatively better if it produces lower-temperature answers more frequently.
        However, recognizing lower-quality answers is also a sign of intelligence.
        So, the extent to which copycat provides poor answers at low frequency and low desirability could be accounted for as well, even though copycat isn't explicitly told to do this.
       
\section{Results}
    \subsection{Cross $\chi^2$ Table}
        The below table summarizes the results of comparing each copycat-variant's distribution with each other copycat-variant.
        [Insert cross $\chi^2$ table]
\section{Discussion}
    \subsection{Distributed Computation Accuracy}
        [Summary of introduction, elaboration based on results]
    \subsection{Prediction}
        Even though imperative, serial, centralized code is turing complete just like functional, parallel, distributed code, I predict that the most progressive cognitive architectures of the future will be created using functional programming languages that run distributedly and in true parallel. 

\bibliographystyle{alpha}
\bibliography{sample}

\end{document}
